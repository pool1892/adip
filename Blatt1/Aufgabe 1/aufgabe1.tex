\documentclass{article}
\def\datum{\today}

\usepackage[german]{babel}
\usepackage[utf8]{inputenc}
\usepackage[T1]{fontenc}
\usepackage{lmodern}
\usepackage{amsmath, amssymb, amsthm}

\title{1. \"Ubungsblatt}
\author{}

\begin{document}
\maketitle

\textbf{Aufgabe 1:}\\

Ein- und Ausgabespezifikation:
\begin{itemize}
	\item Input: $i \in\mathbb{N}$
	\item Output: n-Tupel $(x_1,\dots,x_n)$, mit $x\in\mathbb{N}$
\end{itemize}

Algorithmus:
\begin{itemize}
	\item[1.] Erstelle $P$ (Menge aller Primzahlen)
	\item[2.] Prüfe $i\in P$
	\begin{itemize}
		\item[i.] $i\in P$, f\"uge i zu Tupel hinzu, dann Schritt 5
		\item[ii.] $i\notin P$, gehe zu Schritt 3
	\end{itemize}
	\item[3.] Teile $i$ durch das kleinste Element $p$ von $P$
	\begin{itemize}
		\item[i.] $i\ mod\ p=0$, f\"uge p zum Tupel hinzu, gehe zu Schritt 4
		\item[ii.] $i\ mod\ p\neq 0$, entferne $p$ von $P$, gehe zu Schritt 4
	\end{itemize}
	\item[4.] Wiederhole Schritt 3 bis $P={\emptyset}$, dann Schritt 5
	\item[5.] Ausgabe des Tupels
\end{itemize}
\end{document}